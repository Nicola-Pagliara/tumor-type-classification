\begin{abstract}
L'analisi differenziale è la parte più significativa dell'analisi delle RNA-Seq. 
I metodi convenzionali di solito mettono in corrispondenza i sample tumorali con
i sample normali provenienti dalla stessa tipologia di tumore. Tali metodi, però,
potrebbero fallire nel differenziare i diversi tipi di timore siccome non possono
sfruttare la conoscenza proveniente da altre tipologie di tumore. Il Pan-Cancer
Atlas fornisce informazioni molto ampie sulle 33 classi prevalenti di tumori che
possono essere utilizzate come conoscenza base per generare biomarker specifici
a seconda della classe tumorale. In questo lavoro abbiamo dapprima replicato
il lavoro svolto nel paper \cite{lyu2018deep} inglobando i dati ad alta dimensionalità
delle RNA-Seq in immagini 2D al fine di usare una rete neurale convoluzionale
(CNN) per eseguire la classificazione delle 33 classi tumorali. 
L'accuracy finale ottenuta è stata del $94.74\%$. 
Abbiamo poi provato, come variante a quanto svolto in \cite{lyu2018deep}, a
modificare i parametri di addestramento della CNN al fine di migliorarne le
performance ma abbiamo ottenuto un'accuracy complessiva del $94.83\%$ che è inferiore
a quanto ottenuto da \cite{lyu2018deep}. Tuttavia abbiamo ottenuto dei miglioramenti
per singola coorte tumorale, come ad esempio ESCA, LGG e LUAD.
In seguito, seguendo quanto fatto, abbiamo sfruttato anche noi l'idea di 
Guided Grad-CAM \cite{selvaraju2017grad} e abbiamo generato, per ogni classe,
le heatmap significative per tutti i geni.
Tramite Pathway Enrichment Analysis, sui geni che hanno mostrato un'alta 
intesità nelle heatmap, è stato possibile validare che tali geni sono correlati
a specifiche path tumorali dimostrando che è possibile utilizzare tale lavoro
per la ricerca di potenziali biomarker.
\end{abstract}

% abbiamo riportato i risultati dell'accuracy GPU
