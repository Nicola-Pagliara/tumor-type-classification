\section{Discussione}
\label{sec:discussion}
La scoperta dei biomarker specifici per ogni tumore è essenziale al fine di migliorare i popolari
test genomici attuali. Con la saliva o con il sangue, questi test posso indicare le possibilità
stimate dei diversi tumori per ogni individuo. L'essenzialità è dovuta al fatto che non è garantito che
i biomarker generati dall'analisi differenziale siano specifici per il determinato tipo di tumore e
dal fatto che alcuni tumori potrebbero condividere gli stessi biomarker.
Per evitare questa problematica, in \cite{lyu2018deep} hanno progettato un metodo che sfrutta la conoscenza
di diversi tipi di tumore e che va a cercare i geni che possono essere usati per differenziarli.
In questo lavoro abbiamo replicato il lavoro di \cite{lyu2018deep} e dunque abbiamo utilizzato due reti
neurali convoluzionali per effettuare la classificazione dei dati genomici. La prima rete usata è quella
progettata da \cite{lyu2018deep}, mentre la seconda è una sua variante ispirata a VGG
\cite{simonyan2014very} il cui scopo era di migliorare le performance generali (purtroppo non riuscendoci) e 
migliorare l'accuracy per le singole coorti tumorali. Tale miglioramento è stato riscontrato, ad esempio, 
per OV e STAD.

La ricerca in ambito di computer vision si è sviluppata velocemente e molti dei metodi sono 
stati progettati per risolvere i problemi usando le deep neural network.
Uno dei problemi presenti è che i dati genomici di solito hanno un'alta dimensionalità mentre
molte delle architetture di deep learning sono per immagini 2D.
In questo lavoro abbiamo mostrato che il metodo sviluppato da \cite{lyu2018deep}, nel quale si
inglobano in maniera naive i dati genomici (i geni) su ogni pixel di un'immagine 2D in base 
all'ordinamento cromosomico, è valido e che le performance sono eccellenti tranne che per alcuni
tipi di tumore. 
In base a tali risulti è auspicabile che molti altri metodi di deep learning
possano essere applicati ai dati genomici in futuro.