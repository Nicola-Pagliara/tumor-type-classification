\section{Introduzione}
\label{sec:intro}
Grazie al miglioramento delle tecniche di sequenziamento con l'avvento dei Next Generation Sequencing methods, 
si è avuto un progressivo miglioramento in accuratezza ed efficienza anche per quanto riguarda l'analisi del genoma
umano. Un problema sempre presente anche ai giorni nostri è quello di comprendere in maniera completa e approfondita
le cause dei vari tumori. A tal fine, The Cancer Genome Atlas, si è occupato del sequenziamento e della gestione di
un grande volume di tessuti tumorali e, inoltre, ha analizzato oltre 11.000 tumori appartenenti alle 33 forme più diffuse
di cancro. A partire da tali analisi è stata incentivata la creazione del Pan-Cancer Atlas, il quale, come riportato
nella loro home page "fornisce una comprensione unica, completa, approfondita e interconnessa di come, dove e 
perché nascono i tumori nell'uomo". Il Pan-Cancer Atlas è di fatto una risorsa essenziale per lo sviluppo di nuovi
trattamenti nella ricerca della medicina personalizzata, in quanto, per ogni sample tumorale, è possibile accedere ai
suoi \textit{RNA-Seq expression data}. Tali dati sono un vantaggio quando si cerca di identificare i potenziali
biomarker per ogni tipo di tumore.

Finora, per tentare di scoprire i potenziali biomarker, molte analisi cercavano di trovare i geni che si sono espressi 
differentemente. Tali analisi, però, non tengono conto dell'espressione dei geni in altri tipi di tumore e solo legate
al tipo di tumore e/o ai dati specifici che si stanno analizzando correntemente. Di fatti, molti modelli vanno a 
simulare l'espressione dei geni ma falliscono miseramente quando si usa il modello su altri dati o su altri
tipi di tumore. Da tali fallimenti è dunque nata l'esigenza di progettare un metodo che possa includere nelle analisi
la conoscenza proveniente da più classi tumorali.

Da un altro punto di vista, la classificazione dei tumori usando i dati genomici può contribuire ad una diagnosi più
veloce e più accurata ma, di fatti, raramente viene usata in quanto è difficile lavorare con dataset genomici ad alta
dimensionalità. 
Per rendere più chiaro questo punto, basti pensare che nel Pan-Cancer Atlas gli \textit{mRNA-Seq gene expression} 
data contengono informazioni a partire da oltre 20 mila geni, molti dei quali non sono coinvolti nello sviluppo dei
tumori. 
Tali geni sono dunque weak feature quando si lavora con tecniche di Machine Learning. Usare metodi classici di ML 
come KNN non è fattibile in quanto essi fallirebbero a causa della cosiddetta "curse of dimension". 
È da notare che, se anche la classificazione dei tumori è ancora agli inizi, la classificazione e il riconoscimento di
immagini è invece molto sviluppata e in tale ambito è usato ampiamente il deep learning. Molte architetture come ResNet
\cite{he2016deep} e Inception \cite{szegedy2015going} hanno mostrato performance eccellenti nella classificazione 
multi-classe di immagini.
Inoltre, per comprendere come lavorano le deep neural network, nel campo della computer vision sono stati sviluppati
metodi come la deep Taylor Decomposition, la layer-wise decomposition e Grad-CAM \cite{selvaraju2017grad}. 
Tali metodi generano heatmap delle immagini in input al fine di indicare quanto ogni pixel ha contribuito alla
classificazione.
Dal momento che i \textit{gene expression data} contengono oltre 10 mila geni è promettente utilizzare le deep neural
network per ottenere una buona accuracy. Allo stesso tempo, se si suppone che l'importanza di un gene sia paragonabile 
a quanto si è espresso (e dunque a quanto ha contribuito alla classificazione), si possono prendere in prestito i 
metodi di interpretazione delle deep neural network per scoprire i top gene di ogni tumore. 
A tale scopo, ad ogni gene sarà associato un confidence score, e i geni con uno score alto saranno considerati come top
gene e dunque potenziali biomarker dal momento che la loro esistenza ha influito in maniera considerevole nella 
classificazione.

A partire dal lavoro svolto da Lyu e Haque \cite{lyu2018deep}, dapprima abbiamo provveduto a creare il dataset 
su cui lavorare a partire dai \textit{raw data}. Poi, abbiamo replicato il loro lavoro andando a filtrare i geni 
nei sample che presentavano una varianza piccola. 
Successivamente, abbiamo incorporato gli expression data ad alta intensità (10381x1) in immagini 2D (102x102) 
per adattarli ai layer convoluzionali. In seguito, abbiamo utilizzato la rete convoluzionale di Lyu e Haque 
unitamente alla 10-fold cross validation per testare le performance. Con la rete addestrata, abbiamo seguito 
al pari degli autori, l'idea della Gradient-weighted Class Activation Mapping (Guided Grad-CAM \cite{selvaraju2017grad})
e sono state generate le heat-map per tutte le classi mostrando i pixel più significativi (geni). 
Tramite Pathway Enrichment Analysis, abbiamo poi validato, come in \cite{lyu2018deep}, che la scelta dei top gene
effettuata è biologicamente significativa per i tumori corrispondenti e ciò dimostra che il lavoro di partenza è valido.
Per agevolarci nel compito abbiamo dapprima effettuato dei test su un numero ristretto di classi tumorali.
Dapprima abbiamo lavorato su un caso binario (classi DLBC e UCS), poi su un caso ternario (BLCA, CESC e LGG) e 
infine sul caso generale (tutte e 33 le classi tumorali).

\subsubsection{Pan-Cancer Atlas}~\newline
The Cancer Genome Atlas (TCGA), un programma di riferimento per la genomica del cancro, ha caratterizzato 
molecolarmente oltre 20.000 campioni primari di cancro e campioni normali appaiati che coprono 33 tipi di cancro. 
Questo sforzo congiunto tra il National Cancer Institute (NCI) e il National Human Genome Research Institute è 
iniziato nel 2006 riunendo ricercatori di diverse discipline e molteplici istituzioni. 
The Cancer Genome Atlas (TCGA) ha contribuito a stabilire l'importanza della genomica del cancro\footnote{Il cancro è 
un gruppo di malattie causate da cambiamenti nel DNA che alterano il comportamento delle cellule, provocando una
crescita incontrollabile a livello molecolare e causando gravi danni al sistema cellulare. Queste anomalie
possono assumere diverse forme, tra cui mutazioni del DNA, riarrangiamenti, soppressioni, amplificazioni e aggiunta 
o rimozione di marcature chimiche.
Questi cambiamenti possono indurre le cellule a produrre quantità anomale di particolari proteine o a produrre proteine
malformate che non funzionano come dovrebbero. Spesso, una combinazione di diverse alterazioni genomiche portano alla
formazione di un cancro. Per risolvere  e capire al meglio questi cambiamenti genomici si utilizzano dati clinici che
descrivono la risposta dei pazienti al trattamento del cancro, esperimenti di laboratorio che utilizzano linee cellulari
e organismi modello e tecniche di analisi dei big data.}, ha trasformato la nostra comprensione del cancro e ha persino
iniziato a cambiare il modo in cui la malattia viene trattata clinicamente. L'impatto va ancora oltre poi, raggiungendo
le tecnologie sanitarie e scientifiche, la biologia computazionale e altri campi di ricerca. Dopo 12 anni, con i
contributi di oltre 11.000 pazienti e l'incredibile impegno di migliaia di ricercatori, il TCGA ha prodotto una serie 
di dati di incommensurabile valore. Questi dati\footnote{2.5 petabytes} rimangono a disposizione del pubblico 
come riferimento affidabile che potrà essere sfruttato per molti anni. 
Da questi sforzi nasce il Pan-Cancer Atlas, una raccolta di analisi trasversali sul cancro che ne approfondiscono 
temi generali, tra cui: i modelli di origine cellulare, i processi oncogenici e le signaling pathway.
Grazie all'analisi di oltre 11.000 tumori provenienti da 33 delle forme tumorali più diffuse, il Pan-Cancer Atlas
fornisce una comprensione unica, completa, approfondita e interconnessa di come, dove e perché nascono i tumori
nell'uomo. Come punto di riferimento unico e unificato, il Pan-Cancer Atlas è una risorsa essenziale per lo sviluppo 
di nuovi trattamenti nella ricerca della medicina di precisione, il progetto è stato reso pubblico con
tutti i dati dopo il suo completamento nel 2018.

\subsection{Next Generation Sequencing e RNA-Seq}
Come affermato in \cite{illumina2021rna} con le tecniche Next Generation Sequencing e quelle NGS-based RNA sequencing 
(RNA-Seq) si è avuto un progresso tecnologico che ha consentito agli scienziati 
di andare oltre quelli che erano i limiti dei metodi tradizionali. 
Inizialmente i ricercatori guardavano il 90\% del genoma umano unicamente come 
"junk DNA" mentre ora esso è apprezzato per il suo ruolo di controllo sui geni 
espressi includendo anche informazioni quali: le regioni in cui si sono espressi,
in quale momento e in che entità.
Grazie ai GWAS (Genome-Wide Associations Studies) è stato rilevato che la 
maggior parte delle varianti identificate sono presenti nelle regioni 
non-codificanti del DNA e ciò sottolinea quanto sia importante la \textit{gene expression e regulation} 
nel meccanismo di una malattia.
RNA-Seq rivela il trascrittoma completo (full transcriptome), non solo pochi 
trascritti selezionati. L’RNA-Seq fornisce visibilità all’interno di cambiamenti
nella gene expression che prima non erano rilevabili e consente la 
caratterizzazione di forme multiple di RNA non codificante.
Grazie all'autentico potere di scoperta del rilevamento imparziale dell'RNA, 
l'RNA-Seq è rapidamente emerso come l'approccio più importante per la 
profilazione del trascrittoma ad alto throughput e attualmente è uno 
dei tool più potenti e significativi.

L'analisi dell'espressione genica con RNA-Seq è considerata uno strumento 
fondamentale per scoprire i meccanismi del cancro e aiutare la ricerca sulle 
malattie genetiche. L'RNA-Seq fornisce anche una visione dei trascritti non 
codificanti e ne illumina il ruolo nelle malattie complesse.
Gli avanzamenti nell'RNA-Seq hanno permesso ai ricercatori di esaminare i 
dettagli dello sviluppo del cancro e delle malattie infettive a livello di 
singola cellula con un contesto a livello di tessuto.

\subsection{RNA-Seq vs Microarray DNA}
L’RNA-Seq è un potente metodo basato sul sequenziamento che cattura uno spettro
completo e informativo dei dati di espressione genica. A differenza dei microarray,
RNA-Seq non richiede l'uso di sonde (probe) predefinite e dunque i set di dati
sono imparziali e consentono una progettazione sperimentale priva di ipotesi.
Tale strumento si rivela fondamentale negli studi di \textit{transcript discovery} e di
varianti che non sarebbero possibili con l'utilizzo di metodi tradizionali che
prevedono che l'esperimento abbia un target noto. 

L'RNA-Seq offre una copertura più fine del trascrittoma e una minore 
variabilità tecnica rispetto ai microarray, con la capacità di rilevare una 
percentuale più elevata di geni espressi in modo differenziato, in particolare 
sui geni a bassa abbondanza.

\subsection{Funzionamento di RNA-Seq}
L'RNA-Seq conta le singole \textit{sequence read} allineate ad una \textit{reference sequence} per
generare conteggi discreti di read. I ricercatori, aumentando o diminuendo
il numero di sequence read (\textit{coverage depth}), possono regolare con precisione
la sensibilità di un esperimento al fine di soddisfare i diversi obiettivi di studio.
La natura quantitativa di questo processo e la capacità di controllare i 
livelli di copertura supportano un intervallo dinamico estremamente ampio, 
con valori di espressione assoluti piuttosto che relativi.

\subsection{RNA-Seq per la ricerca sul cancro}
L'RNA-Seq si è rivelato uno strumento essenziale per misurare in maniera diretta
le conseguenze funzionali delle mutazioni dei geni. Dalle 46 mutazioni circa contenute
nel cancro medio, ne servono solo da 5 a 8 per dare inizio al tumore.
Il solo profilo genomico è insufficiente per differenziare queste "driver mutation"
dalle "passenger mutation", ovvero quelle mutazioni che non influenzano l'inizio
o la progressione del cancro. Combinando le misurazioni dei modelli di espressione
genica e le conseguenze delle mutazioni tramite RNA-Seq è possibile differenziare,
su larga scala e in maniera imparziale, i fattori cruciali per la progressione
del cancro e ciò consente una modellazione più approfondita e accurata dello stesso.
Il rilevamento delle fusioni geniche è particolarmente significativo per la 
ricerca sul cancro, poiché il 20\% di tutti i tumori umani presenta 
traslocazioni e fusioni geniche. La maggior parte delle fusioni geniche ha un 
impatto significativo sulla tumorgenesi e una forte associazione con il 
fenotipo morfologico, rendendole utili come potenziali marker diagnostici e 
prognostici.

\vspace{5mm}
La parte restante di questa relazione è organizzata come segue: nella sezione 2 vengono passati 
a rassegna i lavori correlati alla classificazione dei tumori e alla visualizzazione delle deep neural network. 
Nella sezione 3 vengono descritti i dati utilizzati, le metodologie applicate per la classificazione dei 
tumori e la procedura per generare le heatmap. 
Nella sezione 4 vengono discussi i risultati ottenuti e la validazione tramite 
Pathway Enrichment Analysis dei top gene estrapolati dalle heatmap. Infine, nella sezione 5 presentiamo
quella che è stata la nostra modifica al lavoro di Lyu e Haque \cite{lyu2018deep}.