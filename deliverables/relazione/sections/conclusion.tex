\section{Conclusioni e Sviluppi Futuri}
\label{sec:conclusion}
In questo lavoro ci eravamo posti come obiettivo quello di replicare nella sua interezza il lavoro svolto
in \cite{lyu2018deep} e di provare a migliorarlo andando ad agire sulle varie parti che lo compongono.
Dapprima ci siamo occupati di scrivere le parti di codice mancanti (tutte le modifiche sono documentate 
nell'Appendice \ref{appendix}), poi abbiamo eseguito dei test sia su casi ridotti sia sul caso completo 
e infine abbiamo provato a migliorare le performance utilizzando un altra rete neurale come 
descritto nella sezione \ref{sec:varnet}.
È stato un lavoro lungo e difficile in quanto è stato necessario scrivere ex novo alcune parti di codice 
ed è stato necessario aggiornare il codice per meglio sfruttare i recenti miglioramenti nelle librerie
utilizzate.
In conclusione, seppur utilizzando la stessa rete del riferimento all'inizio, abbiamo ottenuto sia delle
performance generali migliori sia performance migliori per alcune coorti tumorali specifiche. 
Inoltre, questo lavoro è stato fonte di ispirazione per lo studio ed è stato fondamentale per comprendere
quanto il lavoro dei bioinformatici possa essere d'aiuto alla ricerca medica senza mai ovviamente 
poterla sostituire.
Sviluppi futuri di questo lavoro sicuramente potranno migliorare le performance magari 
utilizzando un modello di deep learning diverso (ad esempio utilizzando modelli pretrained come VGG, 
AlexNet o Inception) oppure utilizzando reinforcement learning o i transformers.  
Un'altra strada percorribile è quella di fare un tuning ancora più preciso di quelli che sono stati i parametri
di addestramento utilizzati al fine di trovare una configurazione migliore di quella attuale.